\textit{Infinite growth in a finite world is impossible}. We all heard this sentence once in our lifetimes and it is important to remind us of it from time to time when we try to use the resources available to us. When we design a project like this it is not only important to estimate the direct benefits it will have but also what are the implications in the long term or on remote places.
IT is known for its many technological resources and potential impact on the life of everyone but the concept of \textit{social justice} must not be forgotten and it needs to be taken into account during the design and implementation phase of any project to not create more inequality or borrow resources that will never get put back in place, further expanding the breach between poor and rich areas or creating division within communities.

As engineers we have the responsibility of creating systems that relieve people from tasks and improve their lives and try to minimize the negative effects or even reconsider the whole project if those outweigh the benefits.

\subsection{Environmental Sustainability}
The proposed implementation of the project uses relatively few resources. Apart from a workstation --something that most projects nowadays require-- a few small boards are needed. However, it's worth noting that this tools have a significant impact due to the ores needed to manufacture silicon and the ways of extraction as well as the transportation of the devices across the world for assembly and distribution --for example, the ESP-Prog board was shipped from China in 10 days--. On the other hand, because I use my own computer I get to decide where to source it from and in this case it's second hand, but this is not applicable in a business setting where buying equipment second hand is rarely done.

The project strives to make telecommunication more efficient which should, in turn reduce the resources needed to provide connectivity to remote nodes that currently use more resources, be it for energy consumption or infrastructure that could be superseded by the knowledge of the research. However, we need to acknowledge the Jevons' paradox\cite{Alcott2005} which states that improvements in efficiency do not imply a reduced use of the resource but totally the opposite.

\subsection{Economical Sustainability}
The estimated cost of the complete process of the project is estimated to be 16644\euro\ as it will be seen in \autoref{sec:Total}.

Currently the field of LoRa Mesh is in its infancy and with great potential for development and optimization which makes this project an interesting topic with great opportunity for a return of investment as current implementations either use some technology like GSM, which is expensive and power requiring or, in case they use LoRaWan, the costs of setting up and maintaining the gateways significantly increases the costs of the installations. This research will try to put some light into mesh networking, specially for LoRa devices and make it open to the public so anyone can benefit from it or research on top of it.

\subsection{Social Sustainability}
I think this project is a great opportunity to get introduced to the research world at the same time I work with something that I have interest in. Because I have a great amount of flexibility, I expect this to give me the expertise in project management and problem-solving skills when facing issues during the project as well as a good insight in LoRa and mesh networking.

The project will have real implications in the field of telecommunications in remote areas where infrastructure is expensive to set up and maintain, thus allowing entities to save on costs, either economical or human, by automating some or all the tasks of certain processes, as presented in \cite{LoRaMoto}. Hopefully this research will help to save millions of euros in infrastructure and thousands of man-hours by providing --albeit small-- connectivity to remote or challenging places.