 	% Fer servir una llibreria de logging que digui el temps en el que s'ha imprés el missatge (posar una cita a l'article del rellotge atomic de FB que van fer fa poc per demostrar que és una tasca díficil, la de sincronitzar diferents dispositius) i mencionar que es podria fer servir la llibreria desenvolupada en un principi per a sincronitzar els nodes per a fer proves per a obtenir el temps (ja que si s'inicien en monents diferents no estaran sincronitzats)
    % Acabar el featureset de la llibreria
    	% Que pugui enviar i rebre missatges personalitzats per l'usuari
    % Afegir una cua de missatges per no incumplir les regulacions de dutycicle i que els missatges s'enviïn automaticament
    % Afegir una altra cua amb més prioritat per missatges de control
    % Fer piggybacking de missatges reals per transmetre missatges de control
    % Fer estudi estadísic
    	% Mirar quin és el període òptim de l'enviament de missatges de control i si convé que aquest perídoe canviï amb el temps
    	% Estudiar les capacitats de la llibreria sota diferentes topologies
    	% Fer servir metriques (Posició, nivell de bateria, nombre d'enllaços) per a cambiar diferents paràmetres de la ràdio (freqüència, SF, bandwith, potencia)
    % Ampliar la configurabilitat de la llibreria. Diferents casos d'ús poden necessitar més o menys features
    % En el llarg termini, desfer-se de RadioLib, que tot i que és una molt bona llibreria té una mica de overhead (mencionar que al afegir la nostra header s'ha de copiar la payload del usuari ja que necessitem passar-li el paquet a RadioLib en una posició de memoria on estigui contigu mentre que la llibreria "Arduino-LoRa" va fent push de la payload una cua hardware i podriem utilitzar la mateixa estratègia

Considering the project has had some hiccups, there is still some work to do. The main issues are related to library improvements to make it more performant and transparent to the user.
% Fer una llista aqui sense punts ni ordres ni res, com el que vas considerar per les tasques