 	% Fer servir una llibreria de logging que digui el temps en el que s'ha imprés el missatge (posar una cita a l'article del rellotge atomic de FB que van fer fa poc per demostrar que és una tasca díficil, la de sincronitzar diferents dispositius) i mencionar que es podria fer servir la llibreria desenvolupada en un principi per a sincronitzar els nodes per a fer proves per a obtenir el temps (ja que si s'inicien en monents diferents no estaran sincronitzats)
    % Acabar el featureset de la llibreria
    	% Que pugui enviar i rebre missatges personalitzats per l'usuari
    % Afegir una cua de missatges per no incumplir les regulacions de dutycicle i que els missatges s'enviïn automaticament
    % Afegir una altra cua amb més prioritat per missatges de control
    % Fer piggybacking de missatges reals per transmetre missatges de control
    % Fer estudi estadísic
    	% Mirar quin és el període òptim de l'enviament de missatges de control i si convé que aquest perídoe canviï amb el temps
    	% Estudiar les capacitats de la llibreria sota diferentes topologies
    	% Fer servir metriques (Posició, nivell de bateria, nombre d'enllaços) per a cambiar diferents paràmetres de la ràdio (freqüència, SF, bandwith, potencia)
    % Ampliar la configurabilitat de la llibreria. Diferents casos d'ús poden necessitar més o menys features
    % En el llarg termini, desfer-se de RadioLib, que tot i que és una molt bona llibreria té una mica de overhead (mencionar que al afegir la nostra header s'ha de copiar la payload del usuari ja que necessitem passar-li el paquet a RadioLib en una posició de memoria on estigui contigu mentre que la llibreria "Arduino-LoRa" va fent push de la payload una cua hardware i podriem utilitzar la mateixa estratègia

Considering the project has had some hiccups, there is still some work to do. The main issues are related to library improvements to make it more performant and transparent to the user.
% Fer una llista aqui sense punts ni ordres ni res, com el que vas considerar per les tasques
\begin{description}
	\item[Finish the featureset] The current implementation still does not allow for the user to send their own payload or set up a routine to treat the incoming packets. There shoudn't be much work related to make this work as the library already has the relevant working code.
	\item[Add packet queues] As mentioned in \ref{ssec:Sending}, the need for queues that will act as a buffer for sending messages is needed if we want to stay compliant with current regulations. This queue would hold the messages to be transmitted (either telemetry messages or messages that use this node as a relay to the final destination). It would also be interesting to consider the option of having two queues, one with more priority than the other.
	\item[Optimize telemetry transmision] The current routine transmits a fixed size array of telemetry that is rarely full, we need to transmit only relevant data.
\end{description}

In addition, the intended research should still take place and an analisis of the library performance should be carried to understant what factors are more significant and how we can tweak them in order to increase performance.
\begin{description}
	\item[Optimal telemetry frequency] More research needs to be put in place to optimize how often telemetry is relayed to other nodes. This library does not focus on mobile networks so once the network has settled there will be little reason to keep sending telemetry data repeatedly and we could wait some minutes between each transmission. In addition, this would also allow us to free more time for data packets.
	\item[Study performance in different topologies] The library should be stable no matter what topology the network has. There needs to be explicit testing in this regard to guarantee this.
	\item[Use more telemetry data for path optimization] Telemetry such as battery level or current receiving power on neighbouring nodes should be used to set and continuously change the parameters of the LoRa trasnmission to optimize the network performance. If one node is well connected but it is likely to run out of battery, the network should make the decision of not using this node to relay messages if there are alternative paths, for example.
\end{description}