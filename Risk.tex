Every project has expected and unexpected issues and it's important to plan ahead to make sure certain circumstances affect the least amount to the correct development of the research, thus we need to foresee those circumstances and take the necessary measures for them to not happen or to be able to continue without a big delay. The following is a non-exhaustive list of the problems that may arise:
\subsection{Time predictions and objective overshooting}
Having too complex objectives may delay the end date of the project, not reaching the proposed deadline. Thus, it's important to have flexible and modular objectives that can be added or removed without great impact to the quality of the research but that at the same time allow the deadline to be reached. This method works both ways and if there is too much time left, new \textit{subobjectives} can be added so the results cover a wider range of parameters.

\subsection{Unexpected implementation delays}
Software development is known for taking longer than originally planned. It's paramount to have good coding practices, such as unit testing, using version control tools how they are supposed to be used and having good code quality overall trying to avoid code smells. In addition, the board ESP-Prog will allow to debug the code in the microcontroller as if it was running on our workstation, something that will speed up debugging time significantly.

\subsection{Bad climate conditions}
This project has a time critical task that will need to be performed on specific locations. 
The current situation with CoVid-19 makes planning for in-person events much more difficult as it's not certain what the restrictions will be in some weeks. Thus the testing part of the project can get affected by those restrictions. It's important, to plan ahead and be flexible with the locations and the amount of tests that are planned to be done and consider doing the remaining tests later on.
