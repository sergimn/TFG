\begin{figure}
		\selectlanguage{english}
		\begin{abstract}
			\thispagestyle{plain}
Even though as a society of the XXI century we assume telecomunications are ubiquitous, the truth is that in remote places, the costs on building and running a tradicitonal infraestructure to give connectivity are so high that rarely have economical sense and we need to look for alternatives. Usually, this alternatives have some weaknesses that need to be taken into account depending on the use of the infraestrucure. In this project, we will design and implement a mesh networking library for LoRa, a radio technology that allows for transmissions of long range with a small energy footprint, but with low bitrate and we will explore the difficulties through the development and the possible solutions.			
			
			
			\end{abstract}
	\end{figure}
%\renewcommand*\catalanabstractname{Resum}
\begin{figure}
		\selectlanguage{catalan}
		\begin{abstract}
			\thispagestyle{plain}
			
Tot i que com a societat del segle XXI obviem que les telecomunicacions són per tot arreu, el cert és que en indrets allunyats de la civilització, els costos de construir la infraestructura tradicional per donar connectivitat són tant elevats que no tenen sentit econòmic i s'ha de recórrer a alternatives. Normalment aquestes alternatives tenen alguns punts febles que s'han de tenir en compte depenent de quin ús se'n farà. En aquest treball, dissenyarem i implementarem una llibreria de telecomunicacions en xarxa per a LoRa, una tecnologia de radio que permet transmissions de llarg abast amb un consum d'energia petit, però amb molt poc bitrate i explorarem les dificultats del camí i les possibles solucions.
		\end{abstract}
	\end{figure}
