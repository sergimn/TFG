As mentioned before, there are some use cases where a long range low power mesh network is the most viable solution. However, even though this technology has been growing in the past years, there isn't a flexible implementation that is user friendly, moreso when it comes to a mesh protocol that doesn't rely on flooding the network with the message so the reciever gets the packet\cite{MeshtasticProto} rather than try to optimize the multi-hop protocol so the data only gets relied towards the direction of the final node. Thus, Roger's work will be one of the firsts to publicly design such protocol and with my help we will develop an implementation that satisfies such needs.


%In this cases we need to  change the paradigm and use a network that is able to use all the nodes it requires to transmit data in order to be reliable and cost effective, or what's called a \textit{mesh network}, where there is no central node governing all the other nodes, but it's every node that takes autonomous decisions to make sure it sends the data packets in the correct direction\footnote{Note that we are not talking about physical direction but more \textit{make good use of the resources available} direction}
